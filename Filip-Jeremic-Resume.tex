\documentclass[margin,line]{res}
\usepackage{hyperref}
\hypersetup{colorlinks=true,linkcolor=blue,filecolor=magenta,urlcolor=cyan}
\evensidemargin -.3in
\oddsidemargin -.3in
\itemsep=0in
\parsep=0in
\textwidth=5.5in
\textheight=720pt
 
\urlstyle{same}

\begin{document}
\name{Filip Jeremic \vspace*{.1in}}

\begin{resume}
\section{\sc Contact Information}

\vspace{.05in}

\begin{tabular}{@{}p{3.36in}p{3in}}
\href{mailto:filip@jeremic.ca}{\texttt{filip@jeremic.ca}} & \href{http://jeremic.ca}{\texttt{http://jeremic.ca}} \\
& \href{https://github.com/fjeremic}{\texttt{https://github.com/fjeremic}}
\end{tabular}

\section{\sc Interests}

Compiler development (static and dynamic), programming languages (design and implementation), parallelization, computer
graphics, software protection, reverse engineering, malware analysis, and cryptography.

\section{\sc Technical \newline Skills}

{\bf x86/x86-64 Assembly, z/Architecture Assembly, C, C++, C\#, Java} \\
{\em Expert, 10+ years}

These technologies are the ones which I have used the most over the years. Because of my field of interest, low level
programming languages such as various flavors of assembly, C, and C++ have been my primary languages of choice.
Working on a just-in-time (JIT) compiler for Java has made me deeply familiar with these languages both
from the angle of software engineering and performance optimization.

{\bf Linux, Windows, z/OS, Unix tools } \\
{\em Advanced, 5+ years}

My primary development platforms are Linux and Windows. Most of my current work involves development on non-desktop
architectures so working on a remote machine via SSH is second nature to me. I am very comfortable within a Unix
environment carrying out tasks such as instruction level performance investigations, assembly level debugging, and
remote development using the technologies listed above.

{\bf Docker, Jenkins, CMake, Bash, Python, JavaScript, Perl, HTML, WPF, UWP } \\
{\em Experienced, 3+ years}

These technologies I have used in various ways to aid in my work, however they are not my primary tools and I do not
consider myself an expert in any of them. However, I feel that I am proficient in all of them and often use them to my
advantage when they fit the task at hand.

\section{\sc Professional Experience}
{\bf IBM}, Toronto, Canada

{\em Compiler development - Advisory Software Engineer} \hfill {\bf 2017 - Present}
\vspace{-.12in}

A natural progression of my previous position into a team leadership role within the backend area of the JIT compiler 
technology I had been working on for several years. During this exciting time we worked on open sourcing our compiler
technology as part of the \href{https://github.com/eclipse/openj9}{Eclipse OpenJ9} and 
\href{https://github.com/eclipse/omr}{Eclipse OMR} projects at GitHub. As the open source communities grew, my role has
been evolving into being one of the focal points for cross-platform backend JIT development and community management
withing the realms of my expertise.

{\em Compiler development - Staff Software Engineer} \hfill {\bf 2015 - 2017}
\vspace{-.12in}

An extension of my previous role with a broader focus primarily on performance acceleration of Java workloads on Linux
and z/OS. Among development items this role involved instruction level profiling of Java applications and identifying
bottlenecks which can be optimized in the JIT compiler. The performance investigation is done at an instruction level
but common intermediate language (IL) level optimizations were developed as part of the experimentation such that all
supported backends (x86, Power, z/Architecture, ARM, AArch64) can benefit.

\newpage

Part of my role also involved sharing the burden of some of the team leads responsibilities in technical management of
work items triaged to the rest of the development team, and the delivery of performance improvement targets for the
release of Java 9.

{\em Compiler development - Associate Software Engineer} \hfill {\bf 2013 - 2015}
\vspace{-.12in}

Directly after graduation I took a position at IBM Canada's compiler group where I worked on the backend of the
just-in-time (JIT) compiler for the IBM J9 Java Virtual Machine (JVM) in support of IBM platforms.

\vspace{.10in}

\begin{itemize}
\item Worked on a team developing the backend code generator for z/Architecture based processors
\begin{itemize}
\item Designed and developed various compiler features ranging from intrinsics libraries, register allocation,
instruction scheduling, and optimal instruction selection for the target processor
\item Carried out instruction level performance investigations on various workloads and implemented changes to
the backend code generator to realize the performance gains
\item Postmortem core dump analysis of non-deterministic code generator bugs
\end{itemize}
\item Developed a firm understanding the interaction between the various components of a dynamic runtime 
environment (VM, GC, JIT), particularly in the context of a JVM
\end{itemize}

\section{\sc Patents}

\begin{list}{\labelitemi}{\leftmargin=0cm}
\item \href{https://patents.google.com/patent/US10002010B2}{Patent No. US10002010 B2} \hfill {\bf Issued 2018-06-19}

Multi-byte compressed string representation embodiments define a String class control field identifying compression as
enabled/disabled, and another control field, identifying a decompressed string created when compression enabled. Tests
are noped based on null setting of the compression flag. When arguments to a String class constructor are not
compressible, a decompressed String is created and stringCompressionFlag initialized. Endian-aware helper methods for
reading/writing byte and character values are defined. Enhanced String class constructors, when characters are not
compressible, create a decompressed String, and initialize stringCompressionFlag triggering class load assumptions,
overwriting all nopable patch points. A String object sign bit is set to one for decompressed strings when compression
enabled, and masking/testing this flag bit is noped. Alternative package protected string constructors and operations
are provided. A predetermined location is checked to determine whether supplied arguments to a String class constructor
are compressible is performed.
\end{list}

\section{\sc Publications}

J. Carette, W. M. Farmer, F. Jeremic, V. Maccio, R. O'Connor, and Q. M. Tran, ``The MathScheme Library: Some
Preliminary Experiments", in: A. Asperti, J. H. Davenport, W. M. Farmer, F. Rabe, and J. Urban, eds.,
\emph{Conference on Intelligent Computer Mathematics Work-in-Progress Papers Proceedings}, Technical Report
UBLCS-2011-04, pp.~10--22, University of Bologna, 2011.

\section{\sc Education}

{\bf McMaster University}, Hamilton, Ontario, Canada

\vspace{-.3cm}

{\em Master's Student} \hfill {\bf 2012 - 2013}

\vspace{-.4cm}

M.Eng., Computer Science

{\bf McMaster University}, Hamilton, Ontario, Canada

\vspace{-.3cm}

{\em Undergraduate Student} \hfill {\bf 2008 - 2012}

\vspace{-.4cm}

B.Sc., Honours Mathematics and Computer Science

\begin{center}
\textbf{References available upon request.}
\end{center}

\end{resume}
\end{document}
